%%%%%%%%%%%%%%%%%%%%%%%%%%%%%%%%%%%%%%%%%
% Plasmati Graduate CV
% LaTeX Template
% Version 1.0 (24/3/13)
%
% This template has been downloaded from:
% http://www.LaTeXTemplates.com
%
% Original author:
% Alessandro Plasmati (alessandro.plasmati@gmail.com)
%
% License:
% CC BY-NC-SA 3.0 (http://creativecommons.org/licenses/by-nc-sa/3.0/)
%
% Important note:
% This template needs to be compiled with XeLaTeX.
% The main document font is called Fontin and can be downloaded for free
% from here: http://www.exljbris.com/fontin.html
%
%%%%%%%%%%%%%%%%%%%%%%%%%%%%%%%%%%%%%%%%%

%----------------------------------------------------------------------------------------
%	PACKAGES AND OTHER DOCUMENT CONFIGURATIONS
%----------------------------------------------------------------------------------------

\documentclass[10pt]{article} % Default font size and paper size

\usepackage{fontspec} % For loading fonts
\defaultfontfeatures{Mapping=tex-text}
\setmainfont[SmallCapsFont = Fontin SmallCaps]{Fontin} % Main document font

\usepackage{xunicode,xltxtra,url,parskip} % Formatting packages

\usepackage[usenames,dvipsnames]{xcolor} % Required for specifying custom colors

\usepackage{geometry}
 \geometry{
letterpaper,
 total={170mm,257mm},
 left=20mm,
 top=15mm,
 bottom=10mm
 }
% Margin formatting of the A4 page, an alternative to layaureo can be
% \usepackage{fullpage}
% To reduce the height of the top margin uncomment: 


\usepackage{hyperref} % Required for adding links	and customizing them
\definecolor{linkcolour}{rgb}{0,0.2,0.6} % Link color
\hypersetup{colorlinks,breaklinks,urlcolor=linkcolour,linkcolor=linkcolour} % Set link colors throughout the document

\usepackage{titlesec} % Used to customize the \section command
\titleformat{\section}{\Large\scshape\raggedright}{}{0em}{}[\titlerule] % Text formatting of sections
\titlespacing{\section}{0pt}{3pt}{3pt} % Spacing around sections

% Better formatting for grade tables
\usepackage{array}
\newcolumntype{L}[1]{>{\raggedright\let\newline\\\arraybackslash\hspace{0pt}}m{#1}}
\newcolumntype{C}[1]{>{\centering\let\newline\\\arraybackslash\hspace{0pt}}m{#1}}
\newcolumntype{R}[1]{>{\raggedleft\let\newline\\\arraybackslash\hspace{0pt}}m{#1}}

\usepackage{slantsc}

\begin{document}

\newcommand{\com}[1]{}

\pagestyle{empty} % Removes page numbering

\font\fb=''[cmr10]'' % Change the font of the \LaTeX command under the skills section

\par{\centering{\Huge Seth W. \textsc{Musser}}\bigskip\par} % Your name


%----------------------------------------------------------------------------------------
%	NAME AND CONTACT INFORMATION
%---------------------------------------------------------------------------------------

\begin{tabular}[t]{@{}l} 
  77 Massachusetts Ave, Room 6C-205\\
  Cambridge, MA 02139
\end{tabular}
\hfill% move it to the right
\begin{tabular}[t]{l@{}}
	\href{https://swmusser.com}{swmusser.com}\\
	\href{mailto:swmusser@mit.edu}{swmusser@mit.edu}
\end{tabular}

%------------------------------------------------------------------------
% STATEMENT OF PURPOSE
%------------------------------------------------------------------------

\section{Statement of Purpose}

My goal as a physicist is to probe the role that topology plays in fundamental physics phenomena from the quantum Hall effect to vortex dynamics in a superfluid.  As such, in graduate school I plan to study the role of topology in hard condensed matter.  I am also passionate about helping students to think about physics pictorially and to appreciate and take advantage of symmetry in their work.


%----------------------------------------------------------------------------------------
%	EDUCATION
%----------------------------------------------------------------------------------------

\section{Education}

\begin{tabular}{rl}	
2018 - Present & PhD student in \textsc{Physics}\\
& \textbf{Massachusetts Institute of Technology}: Cambridge, MA\\
\ \\
2017-2018 & MASt (MSc equivalent) with Distinction in \textsc{Applied Mathematics}\\
 &\textbf{The University of Cambridge}: Cambridge, UK\\
\ \\
2013-2017 & BA with Honors in \textsc{Physics}\\
& BS with Honors in \textsc{Mathematics}\\
&\textbf{The University of Chicago}: Chicago, IL\\
&\normalsize \textsc{Cumulative GPA}: 3.97/4.00 \\

\end{tabular}


%----------------------------------------------------------------------------------------
%	SCHOLARSHIPS AND ADDITIONAL INFO
%----------------------------------------------------------------------------------------

\section{Honors and Awards}

\begin{tabular}{rl}
2019-2022 & NSF Graduate Fellow \\

2017-2018 & Churchill Scholar \\

Summer 2017 & Enrico Fermi Institute Undergraduate Research Award\\

Summer 2017 & James Franck Institute Undergraduate Research Award\\

May 2017 & John H Lewis Prize for best graduating physics student\\

Summer 2016& Selove Prize for Summer Research\\

May 2016& Phi Beta Kappa (3$^{\mathrm{rd}}$ year) \\

March 2016 & Goldwater Scholar\\

\com{\textsc{Mar.} 2016 & University of Chicago Nominee for the Astronaut Scholarship\\}

\com{\textsc{Sept.} 2013 & University of Chicago Merit Scholarship\\}

\end{tabular}


%----------------------------------------------------------------------------------------
%	PUBLICATIONS AND PRESENTATIONS
%----------------------------------------------------------------------------------------

\section{Publications and Presentations}

\begin{tabular}{rl}

In preparation with Prof. Irvine:& ``Tunable Nucleation of Superfluid Vortices from Hydrofoil"\\

Talk for ChuSOARS:& ``Vortex Nucleation in Superfluids"\\

Paper for 2016 REU:& \href{http://math.uchicago.edu/~may/REU2016/REUPapers/Musser.pdf}{``Weyl's Law on Riemannian Manifolds"}\\

Paper for 2015 REU:& \href{http://math.uchicago.edu/~may/REUDOCS/Musser.pdf}{``From Hamiltonian Systems to Poisson Geometry"}\\

Talk for 2015 REU:& ``Poisson Geometry with Applications to the Hamiltonian Formulation\\
&of Inviscid Fluid Mechanics"\\

Paper for 2014 REU:& \href{http://math.uchicago.edu/~may/REU2014/REUPapers/Musser.pdf}{``Weakly Nonlinear Oscillations with Analytic Forcing"}\\
\end{tabular}


%----------------------------------------------------------------------------------------
%	RESEARCH EXPERIENCE 
%----------------------------------------------------------------------------------------

\section{Research Experience}
\begin{tabular}{r|p{11cm}}

%------------------------------------------------

\textsc{Apr. 2016 - Present} & University of Chicago Department of Physics \\
\com{Hours/Week: }& Superfluids Researcher\\
\com{Summer 2016: 30-50}&\textsc{PI}: Professor William Irvine\\
\com{Academic Year: 10-20}&\begin{itemize}
\item \footnotesize{Built from scratch simulation of dragging hydrofoil through a 2D superfluid governed by Gross–Pitaevskii equation (GPE); later independently ported to GPU} 
\item \footnotesize{Using simulation to understand the role circulation plays in vortex nucleation, and similarities between superfluid and ideal fluid flow}
\item \footnotesize{Working on a paper detailing controlled nucleation of vortices in a superfluid, using a hydrofoil potential}
\end{itemize}\\
\multicolumn{2}{c}{} \\
\end{tabular}

%------------------------------------------------


\begin{tabular}{r|p{11cm}}
\textsc{Summer 2016, `15, `14}& University of Chicago Department of Mathematics \\
\com{Hours/Week: }& REU Student\\
\com{20-30 }&\textsc{Mentors}: Sean Howe and Yun Cheng, Clark Butler, and Ben Seeger\\
& \begin{itemize}
\item \footnotesize{2016 - Studied Riemannian geometry and the spectrum of the geometric Laplacian to understand Weyl's law and DeWitt expansion}
\item \footnotesize{2015 - Studied Poisson manifolds to develop a rigorous background for understanding the Hamiltonian formulation of inviscid fluid mechanics} 
\item \footnotesize{2014 - Studied the failure of regular perturbation theory to address the weakly nonlinear oscillator and demonstrated two-timing as an alternative approach} 
\end{itemize}\\
\multicolumn{2}{c}{} \\

%------------------------------------------------

\textsc{Jan. 2015 - Mar. 2016} & University of Chicago Department of Mathematics \\
\com{Hours/Week: }& Mathematical Fluid Dynamics Researcher\\
\com{Summer 2015: 20-40}&\textsc{Mentor}: Professor Norman Lebovitz\\
\com{Academic Year: 5-10}& \begin{itemize}
\item \footnotesize{Studied turbulence through seminal papers and texts}
\item \footnotesize{Studied the application of Hamiltonian formulation of inviscid fluid mechanics to stability results for Riemann ellipsoids}
\item \footnotesize{Numerically and analytically evaluated various methods to probe stability within this context} 
\end{itemize}\\
\multicolumn{2}{c}{} \\

\end{tabular}



%----------------------------------------------------------------------------------------
%	WORK EXPERIENCE 
%----------------------------------------------------------------------------------------

\section{Work Experience}
\begin{tabular}{r|p{11cm}}

%------------------------------------------------

\textsc{Sep. 2014 - June 2017} & University of Chicago Department of Mathematics \\
& Junior Tutor for MATH 13000s\\
\com{Hours/Week: }&\begin{itemize}
\item \footnotesize{Lead 80-minute tutorial sessions twice a week to solidify students' understanding}  
\item \footnotesize{Gave quizzes and other formative assessments, and graded homework}
\end{itemize}\\
\multicolumn{2}{c}{} \\

%------------------------------------------------

\textsc{Aug. 2014 -  Oct. 2014} & University of Chicago College Programming Office \\
\com{Hours/Week: }& Orientation Leader\\
\com{5 }&\begin{itemize}
\item \footnotesize{Helped set up and organize events for the Class of 2018's Orientation Week}
\item \footnotesize{Led a group of 30 members of the Class of 2018 in discussions about drugs, alcohol, sexuality, race, and privilege at the college}
\item \footnotesize{Conveyed the college's expectations for behavior while facilitating discussions}
\end{itemize}\\
\multicolumn{2}{c}{} \\
\end{tabular}


%----------------------------------------------------------------------------------------
%	COMPUTER SKILLS 
%----------------------------------------------------------------------------------------

\section{Programming Languages}

\begin{tabular}{rl}
Fluent: &  {\fb \LaTeX}, python, Mathematica, \textsc{Fortran}, OpenCL \setmainfont[SmallCapsFont=Fontin SmallCaps]{Fontin-Regular}\\

Some Experience: & CUDA, C, Lab\textsc{VIEW}\\
\end{tabular}


%----------------------------------------------------------------------------------------
%	INTERESTS AND ACTIVITIES
%----------------------------------------------------------------------------------------

%\section{Extracurricular Activities}

%Vortex Dynamics Journal Club\\
%Fluids Group Journal Club\\
%Society of Physics Students\\

%----------------------------------------------------------------------------------------


\com{
%----------------------------------------------------------------------------------------
%	GRADE TABLES
%----------------------------------------------------------------------------------------

\pagebreak

%------------------------------------------------

\par{\centering\Large \hypertarget{grdsm}{Master of Advanced Study in \textsc{Applied Mathematics}}\par}

\vspace{-0.5cm}
\begin{center}
\begin{tabular}{L{7cm} L{5cm} L{4cm}}
\textsc{Course} & \textsc{Professor}&\textsc{Grade}\\ \hline \noalign{\vskip 0.04in}
Statistical Field Theory & David Tong & 92\% \\
Symmetries, Fields and Particles & Nick Dorey & 93\%\\
Quantum Field Theory & Ben Allanach & 95\%\\
Quantum Information Theory & Nilanjana Datta & 68\%\\
Theo. Phys. of Soft Condensed Matter & Mike Cates & 84\%\\
Classical and Quantum Solitons & Nicholas Manton & 78\%\\
Essay & David Tong & 90\%\\
\cline{2-3}
&\textsc{Degree Rating}& Distinction
\end{tabular}
\end{center}
\bigskip


%------------------------------------------------


\par{\centering\Large \hypertarget{grdsp}{Bachelor of Arts with Honors in \textsc{Physics}}\par}

\vspace{-0.5cm}
\begin{center}
\begin{tabular}{L{7cm} L{5cm} L{4cm}}
\textsc{Course} & \textsc{Professor}&\textsc{Grade}\\ \hline \noalign{\vskip 0.04in}
Honors Mechanics & Henry Frisch & A\\
Honors Electricity and Magnetism & Frank Merrit & A\\
Honors Waves, Optics, and Heat & Mark Oreglia & A\\
Modern Physics & Edward Blucher & A\\
Computational Physics & Juan Collar & A\\
Intermediate Mechanics & Kwang-Jer Kim & A\\
Intermediate Electricity and Magnetism I& David Miller & A\\
Intermediate Electricity and Magnetism II & Kathryn Levin & A\\
Quantum Mechanics I \ \ \ \ \ \ \ \ \ & Carlos Wagner  & A\\
Quantum Mechanics II& Carlos Wagner& A\\
Experimental Physics I	& David Schmitz & P\\	
Experimental Physics II	& William Irvine & A\\
Experimental Physics III	& Jonathan Simon &  A\\
Spacetime and Black Holes &Daniel Holz & A\\
Relativistic QM and Intro to String Theory &David Kutasov & A\\
Participation in Research I &William Irvine &A\\
Participation in Research II &William Irvine &A\\
& & \\
\textsc{Graduate Courses}& & \\
\hline \noalign{\vskip 0.04in}
Advanced Mechanics	 & Wendy Zhang & A \\
General Relativity &Emil Martinec & A\\
Statistical Mechanics &Dam Son & A\\
Many Body Theory &Dam Son & A\\
Quantum Field Theory I &Liantao Wang & A\\
Quantum Field Theory II &Liantao Wang & A\\
Quantum Field Theory III &Carlos Wagner & A\\
\cline{2-3}
&\textsc{Major GPA}&\textbf{4.00}
\end{tabular}
\end{center}
\bigskip


%----------------------------------------------------------------------------------------


\par{\centering\Large \hypertarget{grdsm}{Bachelor of Science with Honors in \textsc{Mathematics}}\par}
\vspace{-0.5cm}
\begin{center}
\begin{tabular}{L{7cm} L{5cm} L{4cm}}
\textsc{Course} & \textsc{Professor}&\textsc{Grade}\\ \hline \noalign{\vskip 0.04in}
Honors Calculus I (IBL) & Sarah Ziesler & A\\
Honors Calculus II (IBL) & Sarah Ziesler & A\\
Honors Calculus III (IBL) & Sarah Ziesler & A\\
Honors Analysis I & Wilhelm Schlag & A\\
Honors Analysis II & Panagiotis Souganidis & A\\
Honors Analysis III & Carlos Kenig & A\\
Honors Basic Algebra I& Madhav V. Nori & A\\
Honors Basic Algebra II& Madhav V. Nori & A\\
Honors Basic Algebra III& Matthew Emerton & A-\\
Point Set Topology&  Inna Zakharevich& A\\
Basic Complex Variables&  Charles Smart& A\\
Intro to Algebraic Geometry& Benson Farb& A\\
\cline{2-3}
&\textsc{Major GPA}&\textbf{3.98}
\end{tabular}
\end{center}

}

\end{document}
